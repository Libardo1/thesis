\section{Проучване}
	\subsection{Какво е моделиране?}
		Моделирането е процесът по създаване на модел. Моделът е представяне на съществуващ обект 
		от анализираната система. Той е близък, но по-прост от обекта в реалната система.
		Когато създава модел, анализаторът използва модела за да предскаже промените в системата.
		От една страна, моделът трябва да е близък до реалния обект и притежава неговите свойства.
		От друга страна, трябва да е лесен за разбиране и променяне.
		Добрият модел е добър компромис между простота и реализъм \cite{Anu}
	
		Препоръчва се, усложняването на модела, да става итеративно (iteratively). 
		Важно е, моделът да продължава да бъде верен, през този процес. Някои техники за това включват:
		\begin{itemize}
			\item Симулиране, включваики модела, с познати входни и изходни данни
			\item Пресечено валидиране (cross-validation) \cite{Mahoney}	
		\end{itemize}		 
		
		В зависимост от средствата, използвани за построяването им, моделите биват: физически
		(обекти, процеси и явления, евентуално различни по физическата си природа от оригинала,
		но с аналогични свойства), математически (теории, методи и обекти – функции, уравнения,
		редове и други), информационни (информационни методи, обекти и процеси), компютърни
		(програми, данни и други) и така нататък \cite{Totkov}
		
		Компютърното моделиране на информацията и автоматизирането на
		информационните дейности предполагат въвеждането, изучаването и използването на
		различни модели на информацията за обектите и за дейностите, в които те участват.
		Информационните обекти и информационните процеси са абстрактни модели на
		информацията и информационните дейности, наречени абстрактни (концептуални)
		информационни модели. В компютърната информатика абстрактните информационни
		модели се проектират и създават под формата на алгоритми и структури от данни. \cite{Totkov}
		
	\subsection{Какво е симулация?}
		Симулация е процес при който се извършват различни действия/експерименти върху построения модел, 
		вместо върху реална система. В най-общото си значение, симулация е инструмент за наблюдаване на това
		как една система работи. Симулацията може да се тества под различни условия и за различни времеви диапазони.
		
		Симулация може да се използва когато искаме да:
		
		\begin{itemize}
		\item Намаляне на риска за недостигане на поставен срок		
 		\item Намиране на непредвидени пречки и проблеми
 		\item Оптимизация на поведението на системата
		\end{itemize}
		
		Видове симулации:
		
		\begin{description}
		\item[Непрекъсната симулация(Continuous simulation)] променя състоянието си в неопределени моменти във времето, използват диференциални уравнения
		\item[Дискретно събитийна симулация(Discrete event simulation)] променя състоянието си в определени моменти във времето, използват се събития
		\end{description}
		
	\subsection{Визуални симулации}
		Интересно за поставената цел е, че е необходима направата на визуален компонент за системата.
		По този начин, лесно ще може да бъдат наблюдавани различни движения на обектите. Това може да 
		доведе до по-успешна и лесна оптимизация на цялостната пътна мрежа в града.
		
		Такъв тип симулация налага използването на знания от компютърната графика. 
		Още повече, доближава ни (и дори довежда) до използването на системи за съставяне на компютърни игри. 
		Изненадващо, игрите не са разглеждани задълбочено, или поне тяхната структура и начин на работа, в академичните среди \cite{Holzkorn}.
		
		Визуалната част на системата е критична за това дали системата ще бъде използвана.
		Лесното и удобно използване са важни за всеки софтуерен продукт. 
		Потребителят е този, за който системата ни трябва да се грижи добре. \cite{Microsoft}
		