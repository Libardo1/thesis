\section{Използвани технологии}

	\subsection{Lua}
	
		Lua е малък скриптов език, който кара да се усмихнеш, когато го използваш. 
		Според авторите му той е: \footnote{\url{http://www.lua.org/about.html}} 
		
		\begin{itemize}			
			\item Бърз				
			\item Лек				
			\item Лесен за вграджане (embeddable)				
			\item Доказан				
		\end{itemize}
		
		Той е напълно безплатен за употреба. Разпространява се под 
		\emph{MIT}\footnote{\url{http://www.opensource.org/licenses/mit-license.php}} лиценз. 
		Доказал се е като добър избор за скриптов език за множество комерсиални 
		игри\footnote{\url{http://en.wikipedia.org/wiki/Category:Lua-scripted_video_games}}.
		
		Автори на езика са Roberto Ierusalimschy, Waldemar Celes и Luiz Henrique de Figueiredo. Те разработват езика в
		университета PUC-Rio\footnote{\url{http://www.puc-rio.br/index.html}}, като част от нуждите за тяхната група
		от технологии за компютърна графика. 
		
		Името на езика идва от португалската дума Lua, която означава ``Луна``.
		
		Езикът е разработен като инструмент за вътрешна разработка на софтуер, но по-късно е бил
		използван в няколко комерсиални проекти около света. В момента е най-широко разпространен в
		индустрията отговорна за създаване на игри. \cite{Ierusalimschy}
		
		Lua е смесица между обектно-ориентиран, функционален и програмиране свързано с данни (data-driven development)
		подход към създаването на програми. Има лесен за научаване синтаксис и сравнително малък брой концепции. 				
		
		Имплементиран е като библиотека за програмния език \emph{C}\footnote{\url{http://en.wikipedia.org/wiki/C_(programming_language)}}. 
		Поради тази причина, той няма концепция за``основна програма`` (main program). Нуждае се от приемник (host), който извиква различни
		части от код, написан на \emph{Lua}.
		
		С помощта на \emph{C} функции, \emph{Lua} може да бъде пригоден за работа в различни, непредвидени от авторите му,
		области и споделя вече съществуващия синтаксис.
		
		\emph{Lua} има дългогодишна история, като вграден скриптов език използван в игрите. Производителността му
		е далеч по-добра от други скриптови езици, дори без имплементация на \ac{JIT}. Написан е на
		\emph{ANSI C} и е изключително лесно да се създаде връзка със съществуващи библиотеки на \emph{C} и \emph{C++}.
		Това става чрез указатели към функции, които нямат нужда от скъпа сериализация на параметрите, между \emph{Lua} и \emph{C}.
		
		\emph{Lua} предоставя високо детайлен контрол върху управлението на оперативната памет. Включва се
		постепенен \ac{GC}, който може да бъде настройван или напълно изключван в важни за игрите моменти.
		Всичко споменато по-горе, заедно с възможността за създаване на богата логика, необходима за модерните игри,
		прави \emph{Lua} добър избор дори и за най-тежките екшън игри.		
		
		\subsubsection{Концепции}					
		
			Lua е динамично типизиран език (dynamically typed), което означава, че само стойностите имат тип.			
			
			Има концепция за прихващане на грешките, която много се доближава до тази на изключенията от езици като
			C++ и Java.
			
			Езикът предоставя и съвместни програми (coroutines). Съвместна програма в Lua представлява изпълнение на независима нишка.		
			
			Използването на класове в \emph{Lua} е малко по-интересно в сравнение с езици като \emph{C} и \emph{Java}. 
			То по-скоро напомня на прототипното наследяване в
			\emph{JavaScript}\footnote{\url{http://www.crockford.com/javascript/inheritance.html}}.
			Не се предоставят класове, всичко което разработчика вижда е по-лесен синтаксис за разработката 
			на подобни на обектно-ориентирани системи.
			\emph{Lua} не предоставя възможност за скриване на информация, т.е. всичко е в един ''клас'' е достъпно
			от всеки. 
			
			Възможно е създаването на наследяване, нещо което отново не се поддържа директно от езика. Проста имплементация
			би търсела несъществуващите полета на текущия обект в друг, предварително зададен от програмиста. \cite{Figueiredo}
			Това позволява на всеки разработчик да създаде свой, собствен стил на писане на класове. Това може да е добро и лошо нещо.
			
		\subsubsection{Имплементация}
		
			\emph{Lua} използва една от първите регистрови виртуални машини в своята имплементация. Така се избягва
			необходимостта от неколкократното вмъкване и премахване на стойности, 
			което е необходимо при стек-базираните виртуални машини. Средният брой изпълнени инструкции, при използване
			на регистрови виртуални машини, е с 47\% по-малко от този при стековите такива. \cite{Khan}
			
			За разлика от много други езици, \emph{Lua} не предоставя масиви. Предоставени са ''таблици'', които
			са асоциативни и могат да бъдат използвани подобно на масиви. Езикът използва методи за
			засичане на това дали една таблица е използвана като масив. Ако случаят е такъв, то имплементацията
			на езика създава истински масив, за да оптимизира изпълнението на програмата.
			
			Предоставят се анонимни функции. Те са първокласни конструкции в езика. Използва се новаторски подход,
			които запазва локалните променливи в стека и ги премества, само когато те излезнат от обсега (scope) и
			се използват от вложени функции.
	
	\subsection{Moai}
	
		Lua е прекрасен малък език, но сам по себе си, той не доставя необходимите инструменти
		за лесно създаване на симулационна система. Намиране на библиотека предоставяща по-висока
		абстракция е задължително.
		
		Една такава библиотека е \emph{Moai}\footnote{\url{http://getmoai.com/}} (произнася се Moe-Eye).
		Тя е по-скоро двигател за игри (game engine).
		Разработката й е започнала през 2010г. и е активно разработвана и до днес. 
		Тя е с \emph{отворен код}\footnote{\url{http://github.com/moai/moai-dev}}
		и използва лицензът \emph{CPAL}\footnote{\url{http://www.opensource.org/licenses/cpal_1.0}}, който
		позволява да използваме софтуера за нашите цели. 
		Разработена е от \emph{Zipline Games}\footnote{\url{http://www.ziplinegames.com/}}.						
		
		\emph{Moai} е библиотека за разработване на игри върху множество от платформи, вариращи от десктоп до
		мобилни операционни системи. За основна графична библиотека се използва стандартът 
		\emph{OpenGL}\footnote{\url{http://www.opengl.org/}}, който позволява тази разнообразност.
		Поради добрата си свързаност със \emph{C}, Moai може да постигне производителност, по-висока
		от тази на приложения написани на  \emph{Objective C}
		\footnote{\url{http://en.wikipedia.org/wiki/Objective-C}} за \emph{iPhone}.
		
		\emph{Moai} разрешава много от проблемите с които се сблъскват разработчиците на мултиплатформени 
		игри във всекидневната си работа. Състои се от клиентска рамка (framework), която ускорява разработката
		на игри, и облачно-базирана (cloud-based) сървърна част, която спомага за създаване на игри, играни от 2-ма
		или повече играча.

		Обектният модел на \emph{Moai} е изцяло написан на \emph{C++} и използва префикс ``MOAI``. Тези обекти
		се изчистват от \ac{GC}, също както и тези в \emph{Lua}. За това спомагат и т.нар. контейнери в \emph{Moai},
		които се грижат да не останат неизчистени обекти от паметта.
		
		По време на създаването на \emph{Moai}, авторите са искали да направят библиотека, която да помага много
		за лесното създаване на симулации\footnote{\url{http://getmoai.com/moai-basics-series/moai-basics-part-1.html}}.
		Например, по подразбиране началото на координатната система е в центъра на екрана. Това позволява по-лесното
		позициониране на обекти като цяло, тъй като обектите които са в центъра на екрана са по-важни за наблюдателя.
		
		Основната концепции на \emph{Moai} са \cite{Zipline}:
		
		\begin{itemize}

			\item \emph{Мащабируемост. (Scalable)} Позволява игрите да бъдат играни от милиони играчи, без това
			да пречи на изживяването им. Работата на програмиста е да направи добра игра, за останалото се грижи
			\emph{Moai} 
			
			\item \emph{Мултиплатформеност. (Cross-platform)} Позволява използването на един програмен език за
			създаване на игри на множество платформи. Предоставя единна обвивка, която намаля риска от нуждата за
			писане на платформено-зависим код.
			
			\item \emph{Отвореност. (Open)} Осигурява лесна промяна на платформата, ако нуждите за текущия
			проект не са добре подсигурени от \emph{Moai}. Кодът е отворен и всеки може да се възползва от това.
			
			\item \emph{Бърза разработка. (Rapid development)} Позволява на екипите от разработчици, бързо и лесно, да
			променят и създават нови части, концепции и графики за своите игри. Предоставя се висока абстракция за да не е
			необходимо повторно създаване на вече съществуващи концепции.
			
		\end{itemize}
		
		\subsubsection{Облакът на \emph{Moai}}
		
			Облакът от услуги на \emph{Moai} е платформа за създаване, разполагане (deployment), управление и 
			хоризонтално скалиране на онлайн компоненти за мултиплатформени игри. Голямата стойност, която облакът
			на Moai предоставя, е премахването нуждата за създаване и настройването на частна инфраструктура, която
			да има възможност за обработка на милиони потребители.
			
			Логиката в облакът може да бъде написана на \emph{Lua}, \emph{C Sharp} или \emph{JavaScript}.
			Това позволява на разработчиците да създават иновации в техните игри, вместо да са заключени
			в стандартизирано \ac{API}.
			
			Архитектурата на облака е 3-слойна. Тя предоставя оптимизации за хоризонтална скалируемост във
			всеки слой. Предоставя се възможност за разширяване на системата за добавяне на нови възможности
			и езици за програмиране.
			
			Първият слой е входна точка, която приема множество уеб протоколи (HTTP, WebSockets, Flash Sockets и др) и ги 
			трансформира до единен формат, който е познат на платформата. Това предоставя балансиране на натоварването (load balancing),
			проверка на съобщенията (message validation), събиране на статистики (metrics gathering) и др.
			
			Вторият слой се грижи за събирането на съобщенията и насочването им към необходимите среди на програмните езици.
			Те представляват сигурни кутии (sandboxes), които не пазят състояния и изпълняват кода на разработчиците.
			
			Третият слой представлява механизми за запазване на информация. Облакът на \emph{Moai} използва споделени
			бази от данни.
			
		\subsubsection{Производителност}
		
			Основни проблеми при създаването на игри са ниската производителност, високата консумация на оперативна памет
			и размери на файловете. Производителността в игрите се измерва с разглеждане на кадрите за секунда и други метрики.
			
			Игрите създадени с помощта на \emph{Moai} може да бъдат по-бързи от такива написани на \emph{Objective-C} или \emph{Java}.
			Това се дължи на високо оптимизирания \emph{C} код и употребата на \emph{OpenGL}. Двигателят на \emph{Moai} автоматично
			запазва и минимализира заявките до \emph{OpenGL}. Има вградена система за геометрично буфериране, която гарантираю
			оптималната употреба на графичния хардуер, дори и за много сложни сцени. Графът на сцени, който \emph{Moai} използва,
			е високо технологична имплементация, която намаля обработката във всеки кадър, само до променените обекти в сцената.
			
			Много разработчици споделят, че \emph{Moai} е перфектен избор за бързо разработване на производителни игри.
			Дори такива със сложна логика, показващи стотици хиляди ефекти са лимитирани само от графичния хардуер.
			Двигателят и игровата логика са толкова бързи, че единственото ограничение е самия хардуер. \cite{Zipline}
			
			