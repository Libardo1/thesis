\section{Проучване}

	За решаване на голяма част от проблемите, описани по-горе, се разработват и
	използват \ac{ITS} в държави като Япония, Сингапур, Южна Корея, САЩ и други.
	Употребата им води до значително нарастване на производителността, включително
	намаляване на задръстванията, подобряване на сигурността и удобството
	по време на пътуването \cite{Ezell}
	
	\subsection{\ac{ITS}}
	
	\ac{ITS} черпят своя ''интелект'' от ситуацията на пътя, който наблюдават. Участници в пътната
	обстановка са превозни средства, пешеходци, пътни платна, кръстовища, светофари и други.
	Основната цел на интелигентните системи е да позволи на всеки наблюдаван участник да
	достигне желаната дестинация - бързо, сигурно и удобно.
	
		\subsubsection{Приложения на \ac{ITS}}
		
			\emph{Системи за уведомяване при аварирали превозни средства (Emergency vehicle notification systems)},
			позволяват автоматично или ръчно свързване с оператор, когато се случи инцидент. По време
			на обаждането, освен гласовите данни, се предава информация за времето на инцидента, вида му,
			местоположението, идентификацията на превозното средство и др.
			
	    \emph{Автоматично засичане на нарушения на пътищата (Automatic road enforcement)} е система състояща
	    се от камера и устройство за наблюдаване на превозни средства. Използва се за засичане
	    на нарушители на ограничението на скоростта, преминаване на червен светофар, неправомерно
	    използване на пътни платна, предназначени за автобуси, неправилно изпреварване и др.
	    
	    \emph{Вариращи ограничения на скоростта (Variable speed limits)} се използват, когато
	    има налично задръстване. В такива ситуации се променят минималната и максималната допустима скорост.
	    Най-често се използват при наличие на отсечка от пътя, която се употребява от множество превозни средства.
	    
	  \subsubsection{Основополагащи технологии за \ac{ITS}}
	  	
	  	\emph{Глобална позиционна система (GPS)} устройства се поставят в превозните средства,
	  	за да изчисляват текущото местоположение. Точността е до 10 м. Държави като Холандия и Германия,
	  	я използват за изчисляване на изминатите разстояния от отделните превозни средства. Това
	  	спомага по-точното таксуване при използване на пътищата.
	  	
			\emph{Специални съобщения в малък обсег (DSRC)}	е канал за комуникация, опериращ на 5.8 или 5.9 гигахерцова честота,
			специално разработена за превозни средства. Най-важната особеност на системата е позволяването на двупосочна комуникация
			между превозните средства и апаратурата до пътя. Каналът позволява и директна комуникация между участниците в пътното движение.
			
			\emph{Безжични мрежи (Wireless Networks)} позволяват комуникация на устроиства до няколко стотин метра. Този проблем
			се решава чрез добавяне на възможност за всеки участник в пътното движение да бъде приемник и изпращач на данни. Използва се същата
			технология, която използват безжичните домашни рутери. 
	  
	  	\emph{Мобилна телефония (Mobile Telephony)} се използва от \ac{ITS} за трансфер на информация по
	  	мобилни мрежи от трета и четвърта генерация. Предимство на тези мрежи е наличието им в градовете.
	  	При използване на мобилна мрежа се наблюдават увеличение на изпратените и получените данни,
	  	които трябва да бъдат покрити от мобилните оператори или потребителите. При спешна нужда
	  	за бързо прехвърляне на голямо количество информация, мобилните мрежи може да не покрият изискванията.	  		  	
	    
	  \subsubsection{Преглед на \ac{ITS}}
	  	  	
	  	\subparagraph{\emph{Интелигентна пътно-информационна система (IRIS)}} е проект с отворен код, 
	  	разработен от отдела за транспорт на Минесота, САЩ. 
	  	Използва се за наблюдение и обработване на междущатски и магистрален трафик.	  	
	  	
	  	Нуждата за създаване на IRIS идва от факта, че предишни системи са разработени
	  	с модули, използващи затворен код и висока цена за използването им. \cite{Darter}
	  	
	  	Някои от възможностите на системата са:
	  	\begin{itemize}
	  		\item Наблюдение и контрол на вариращи скоростни ограничения
	  		\item Контрол на пътните платна
	  		\item Засичане на задръствания
	  		\item Автоматизирана система за предупреждения
	  		\item Камери за наблюдение и контрол на трафика	  		
	  	\end{itemize}
	  	
	  	Системата е успешно интегрирана в Минеаполис, САЩ.
	  		  	
	  	\subparagraph{\emph{MITSIMLab}} е проект с отворен код, разработен от Mасачузетския технологичен институт.
	  	Той представлява симулационно-базирана лаборатория, която е разработена за разглеждане на ефектите 
	  	от различни начини за управление на трафика.
	  	
	  	Софтуерът се състои от три основни модула - микроскопичен симулатор на трафик, мениджър на трафика
	  	и \ac{GUI}.
	  	
	  	Системата е успешно интегрирана в Стокхолм, Швеция. Тя се грижи за интелигентната промяна
	  	на сигналите, подавани на светофарите и даване на приоритети при наближаване на автобуси.
	  	
	  	MITSIMLab е избрана, защото предоставя много възможности, които я правят реалистичен
	  	симулатор на обстановката на пътя. Дизайнът на системата е високо модулизиран. Това
	  	предоставя възможност за лесна променя и добавяне на допълнителни елементи при създаване
	  	на различни сценарии. \cite{Rehunathan}
	  
	\subsection{Нужна ли е \ac{ITS}?}
	
		Има няколко модела за създаване на транспортни симулации. За постигане на целите, изложени по-горе,
		ще се съсредоточим върху микроскопични модели, които показват поведението на различните превозни
		средства и симулация на групи от такива. Проучвания показват, че такива модели показват
		реалистични резултати \cite{Nagel}.
		
		На базата на тези познания се отхвърля нуждата от създаване на сложна \ac{ITS}. Това, което
		ни трябва, е начин за достигане на някаква дестинация по бърз, лесен, удобен и безопасен начин.
		Използване на симулация е основен инструмент при анализ на пътен трафик. Още повече, наличието
		на градски транспорт води до значително намаляне на разходите, риска и вредните газове за отделните участници.

	\subsection{Какво е моделиране?}
	
		Моделирането е процесът по създаване на модел. Моделът е представяне на съществуващ обект 
		от анализираната система. Той е близък, но по-прост от обекта в реалната система.
		Когато създава модел, анализаторът използва модела, за да предскаже промените в системата.
		От една страна, моделът трябва да е близък до реалния обект и притежава неговите свойства.
		От друга страна, трябва да е лесен за разбиране и променяне.
		Добрият модел е добър компромис между простота и реализъм \cite{Anu}
	
		Препоръчва се, усложняването на модела да става итеративно (iteratively). 
		Важно е, моделът да продължава да бъде верен през този процес. 		
		Някои техники за това включват:		
		
		\begin{itemize}		
			\item Симулиране, включваики модела, с познати входни и изходни данни			
			\item Пресечено валидиране (cross-validation) \cite{Mahoney}				
		\end{itemize}		 
		
		В зависимост от средствата, използвани за построяването им, моделите биват: физически
		(обекти, процеси и явления, евентуално различни по физическата си природа от оригинала,
		но с аналогични свойства), математически (теории, методи и обекти – функции, уравнения,
		редове и други), информационни (информационни методи, обекти и процеси), компютърни
		(програми, данни и други) и така нататък \cite{Totkov}
		
		Компютърното моделиране на информацията и автоматизирането на
		информационните дейности предполагат въвеждането, изучаването и използването на
		различни модели на информацията за обектите и за дейностите, в които те участват.
		Информационните обекти и информационните процеси са абстрактни модели на
		информацията и информационните дейности, наречени абстрактни (концептуални)
		информационни модели. В компютърната информатика абстрактните информационни
		модели се проектират и създават под формата на алгоритми и структури от данни. \cite{Totkov}
		
		Симулационните компютърни модели представляват интерес за научните и
		стопанските дейности. Компютърният модел дава възможност за разкриване на структурата,
		механизма и закономерностите, на които се подчиняват различни природни феномени или
		проблеми от реалния свят. По този начин е възможно прогнозиране, управление или
		изкуствено възпроизвеждане на тези модели, което намалява средствата, влагани в
		експериментални изследвания. \cite{Iliev}
		
	\subsection{Какво е симулация?}
	
		Симулация е процес, при който се извършват различни действия/експерименти върху построения модел, 
		вместо върху реална система. В най-общото си значение, симулация е инструмент за наблюдаване на това
		как една система работи. Симулацията може да се тества под различни условия и за различни времеви диапазони.
		
		Симулация може да се използва, когато искаме да:
		
		\begin{itemize}		
			\item Намалим риска за недостигане на поставен срок					
 			\item Намираме непредвидени пречки и проблеми 			
 			\item Оптимизираме поведението на системата 			
		\end{itemize}
		
		Видове:
		
		\begin{itemize}
		
			\item \ac{CS} променя състоянието си в неопределени моменти във времето,
			 използва диференциални уравнения
			 
			\item \ac{DES} променя състоянието си в определени моменти във
			 времето, използва събития
			 
		\end{itemize}
		
	\subsection{Визуална симулация}
	
		Интересно за поставената цел е, че е необходима направата на визуален компонент за системата.
		По този начин, лесно ще може да бъдат наблюдавани различни движения на обектите. Това може да 
		доведе до по-успешна и лесна оптимизация на цялостната пътна мрежа в града.
		
		Такъв тип симулация налага използването на знания от компютърната графика. 
		Още повече, доближава ни (и дори довежда) до използването на системи за съставяне на компютърни игри. 
		Изненадващо, игрите не са разглеждани задълбочено, или поне тяхната структура и начин на работа, 
		в академичните среди \cite{Holzkorn}.
		
		Визуалната част на системата е критична за това дали системата ще бъде използвана.
		Лесното и удобно използване са важни за всеки софтуерен продукт. 
		Потребителят е този, за който системата ни трябва да се грижи добре. \cite{Microsoft}
		
		Следва разглеждане на някои от основните и най-добри симулационни 
		системи\footnote{\url{http://en.wikipedia.org/wiki/List_of_computer_simulation_software}}.
		Целта е да се направи най-добър избор на такава, която е в съответствие
		с поставените ни задачи и изисквания.
	
	\subsection{Преглед на системи за създаване на симулация}
	
		\subparagraph{Simplex3} е система за симулации, която може да работи под Windows и Unix операционни системи.
			Тя е универсално приложима, когато имаме дискретни модели, процеси 
			лесно моделируеми с помощта на опашки или транспортни модели.
						
			Позволява лесно и бързо научаване на системата, като това не изисква
			научаването на нов програмен език от високо ниво.			

			Притежава собствен език за създаване на модел, наречен Simplex-MDL. 
			Той позволява описанието на почти всякакъв вид модели. Въпреки това,
			модели, използващи частни диференциални уравнения не могат да се представят
			лесно.
		
			Благодарение на универсалността си, Simplex3 може да се използва за академични цели.
			Лесно може да се приспособи за случаи, в които няма разработен специализиран симулационен
			софтуер. 
		
			От сайта на системата, не става ясно до колко тя е поддържана. 
			Липсва добра документация.\cite{Simplex3}
		
		\subparagraph{AnyLogic} е платена система за общ вид моделиране и симулационен инструмент за 
			дискретни, непрекъснати и хибридни системи.
			
			Системата предоставя \ac{GUI} за създаване на моделите, 
			които допълнително могат да бъдат разширявани с помощта на програмният език Java. 			
			AnyLogic предоставя моделиране чрез UML-базирано обектно-ориентирано моделиране, блок-схеми, 
			диаграми на автомати, диференциални и алгебрични уравнения и други.

				Предоставя възможност за създаване на Java аплети, които позволяват 
			лесно споделяне на симулациите, както и поставянето им в интернет.
	
			Моделиращият език на системата е разширение на UML-RT - голяма колекция от най-добри практики,
			доказали се при моделирането на големи и сложни симулации.
						
			Тя е полезна, когато искаме да разглеждаме симулации относно:
			
			\begin{itemize}
				\item Контролни системи
				\item Производство
				\item Телекомуникации
				\item Обучение
				\item Логистика(Logistics)
				\item Компютърни системи
			\end{itemize}			
			
			AnyLogic е доста обширна и понякога тежка система. Предоставя голям набор от полезни инструменти.
			Тя е \emph{много скъпа}\footnote{Цената на най-евтиното издание на продукта е \EUR{4800}}, 
			което е в разрез с изискванията ни. \cite{AnyLogic}
			
		\subparagraph{Tortuga} е симулационна система с отворен код, разработена от MITRE Corporation
			\footnote{\url{http://www.mitre.org/}} в периода от 2004 до 2006 година.
			Тя е ориентирана изцяло към дискретно-събитийни симулации.
			
			Симулация в системата може да се представи като взаимодействия между процеси или планирани (scheduled) събития.
			Системата изисква множество от написани (от потребителя) класове на Java, инсталиран \ac{JDK}, 
			Ant - инструмент за автоматично билдване (build) и самата симулационна система. 
			Предоставя се и \ac{GUI} за управление и наблюдение на самата симулация.
					
			Tortuga е изцяло написана на Java. Това прави възможно използването й под Windows и Unix базирани системи.
			Интересното тук е, че за постигане на своите цели Tortuga използва AOP (Aspect Oriented Programming)
			\cite{AOP}, като не се изисква от потребителя да има познания в тази насока, стига да компилира 
			програмата си по специфичен начин. 			
		
			Системата използва програмна парадигма, която многократно намаля сложността при създаване на симулация.
			Тя третира всяка единица от симулацията като отделна нишка. Java виртуалната машина ограничава броя на нишките
			\footnote{Броят варира спрямо имплементацията, настройки при пускане и хардуера}, 
			което води до ограничение броя на симулационните единици.
		
 			Tortuga може да бъде използвана самостоятелно или като част от по-голям проект. 
 			Тя предоставя споделяне на симулации в Java аплети, също както и AnyLogic.
 			
 			Лицензът който използва е \ac{LGPL}\footnote{\url{http://www.gnu.org/licenses/lgpl.html}}, 
 			което прави системата добър кандидат за по-нататъшно разглеждане.
 			Възможността й за използване под много платформи (благодарение на Java виртуалната машина) е 
 			още един голям плюс.
 			Не трябва да забравяме използването и на \emph{AOP}, което я изкарва от "стандартна" Java програма.
 			Според сайта на системата, от 2008 година, поддръжката за нея е спряна. \cite{Tortuga}
					
		\subparagraph{SimPy} е система с отворен код, създадена през 2002 г. от Klaus G. Muller и Tony Vignaux. След това
			към тях се присъединяват много други разработчици. 
			Тя предоставя възможност за създаване на дискретно-събитийни симулации.
			
			Симулациите в системата могат да се представят като взаимодействия между процеси. Те се изграждат чрез
			програмен код. За създаването на проста симулация се изискват около 10 реда код, 
			които включват познания по основни класове и функции от системата. Има GUI(използващо библиотеката 
			Tk\footnote{\url{http://www.tcl.tk/}}) за предоставяне на входни данни и наблюдаване на симулациите.						
		
			SimPy е написана на Python\footnote{\url{http://www.python.org/}} и поддържа версии от 2.3 до 3.2 включително.
			Благодарение на Python виртуалната машина (при използване на стандартната CPython 
			имплементация\footnote{\url{http://wiki.python.org/moin/CPython}}, SimPy може да се изпълнява под 
			Windows и Unix операционни системи.
		
			Системата използва обектно-ориентиран подход към създаване на симулации. Благодарение на езика, на който е
			написана, с нея бързо може да се направи дори и по-сложен модел.
								
			SimPy може да бъде използвана самостоятелно, както и навякъде, където има Python интерпретатор.
			Предоставя се набор от пакети за изобразяване на графики и управление на структури от данни, които могат
			да бъдат използвани извън рамките на библиотеката. Готови симулации лесно могат да бъдат пакетирани
			като изпълними файлове и предоставени на други потребители.

				Лицензирана е под \ac{LGPL} лиценз. Системата изглежда добре поддържана. Последната версия е 2.3, излезнала през 
			Декември 2011 г. Използва лесен и достъпен език, но без много допълнителни инструменти за по-бързо и лесно
			създаване на симулация. \cite{SimPy}
				
		\subparagraph{GarlicSim} е безплатен продукт с отворен код\footnote{\url{https://github.com/cool-RR/GarlicSim}}, 
			който се опитва да предостави нова концепция за това как всъщност може да се използват симулациите.
			Автор на системата е Ram Rachum\footnote{\url{http://ram.rachum.com/}}.														
			
			Проектът обещава да напише повтарящата се част от кода, необходим за една симулация, вместо нас и ни
			оставя да работим върху по-важната част от проекта - самата симулация. Самата тя се създава чрез 
			писане на програмен код. Може да постигнем направата на проста симулация с 5 реда код!   			
							
			GarlicSim е изцяло написана на Python и официално все още не поддържа Python 3.x сериите. 
			Системата предлага изчистен и лесен GUI за управление и наблюдение на симулациите. 			
		
			За създаване на системата е използван изцяло обектно-ориентиран подход, който допринася за лесното
			научаване на основните стъпки при правене на симулация. Основна дейност при извършване на симулация
			е т.нар. симулационен пакет (simulation package), който съдържа функция, която определя стъпката за
			дадената симулация.
		
			Според автора, системата е достатъчно обща, за да позволи симуларинето на каквато и да е симулация.
			Дадени възможности са:

				\begin{itemize}
				\item Физични
				\item Теория на игрите
				\item Разпространение на епидемии
				\item Квантова механика
				\item Електрически
			\end{itemize}								
			
			Лицензирана е под \ac{LGPL} лиценз. GarlicSim е един чудесен продукт за начални опити за създаване на симулации. 
			Основен проблем е, че е в алфа версия. Няма сведения за успешни употреби в някаква насока.
			Има добра, но не напълно достатъчна документация. За създаване на собствена симулация е необходимо
			написването на симулационен пакет, което до някъде обезсмисля обещанието за това да пишем по-малко код.
			\cite{GarlicSim}