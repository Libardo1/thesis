\begin{abstract}\thispagestyle{empty}
	Настоящата дипломна работа се състой от въведение, създаване на симулационна система, 
	ръководство за потребителя и заключение.
	
	Във въведението се описва проблема и целта на дипломната работа. За постигане на целта са посочени
	задачи, които трябва да бъдат изпълнени.
	
	Основната част е създаване на симулационна система. Тя съдържа четири основни части:	
	\begin{itemize}
		\item \emph{Проучване} - кратко разглеждане на съществуващи решения и симулационни системи
		\item \emph{Моделиране} - запознаване със същността на моделирането
		\item \emph{Използвани технологии} - преглед на използваните технологии за изграждане на решението
		\item \emph{Реализация} - описва процесът на създаване на системата
	\end{itemize}
	
	Ръководство за потребителя съдържа описание на основните функционалности на системата, как могат
	да се ползват и какво може да се променя. Това е допълнено с екрани от програмата.
	
	В заключението се предоставят резултати от наблюдаваните експерименти и система,
	решаваща поставените задачи. Описват се приносите на автора и проблемите срещнати по време
	на разработката на програмата. Накрая се дават препоръки за възможни подобрения.

	В края се предоставят списъци от библиография, използваните алгоритми, фигури и използвани съкращения.	
\end{abstract}