\chapter{Увод}

	Искали ли сте да прекарате повече време с детето си на закуска, 
	да се приберете по-рано от работа и да прекарате време с любимият човек или да излезнете с приятели? 
	Може и да поспите повече, разбира се. Колко хубаво би било това да е правило, вместо изключение.
	Всеки пловдивчанин, използващ пътната мрежа на града, губи средно около 60 мин. на ден в трафика. 

	\section{Цел}

	Целта на дипломната работа е да предложи начин за намаляне на времето прекарано в пътната мрежа на град Пловдив с 10\%. 

	\section{Изисквания}

		\begin{itemize}
			\item цената за използването на градската мрежа трябва да остане същата или да е по-ниска
			\item повишаване на комфорта и спокойствието при използването на градската мрежа
			\item увеличаване печалбите на превозващите компании
			\item намаляне на вредните емисии във въздуха
			\item намаляне на броя катастрофи
		\end{itemize}

	\section{Постигане на целта}

		За постигане на целите на дипломната работа, спрямо поставените изисквания, се разглеждат - 
	
		\begin{itemize}
			\item модерни технологични решения в подобни ситуации
			\item създаване на високо паралелизирана симулация за намиране на критичните точки от транспортната мрежа и тяхното оптимизиране
			\item за получените резултати се анализират за да се получи частично или пълно решение на проблема
		\end{itemize}

	\section{Целеви групи}
	
		Хора, които използват транспортната мрежа, през най-натоварените часове на денонощието. Важно за всеки от участниците е бързото достигане на съответна точка от града, без това да пречи на личното им здраве и комфорт.
	
	\section{Желан резултат}
	
		Работата може да се сметне за успешна, ако се постигне намаляне на прекараното време в транспортната мрежа с 10\%.
	
	\section{Задачи}
	
		\begin{itemize}
			\item Проучване върху методите за изграждане на симулации
			\item Избор между съществуваща и специализирана за целта система
			\item Избор на програмен език
			\item Програмиране на самата система
			\item Провеждане на симулации
		\end{itemize}
	
	\section{Структура}
	
		Настоящата дипломна работа се състои от:
	
		\begin{description}
			\item[Увод] обосновка на проблема, поставяне на конкретна цел, целеви групи и желан резултат
			\item[Изисквания]
			\item[Изложение] разделено на три основни части:
			\begin{description}			
				\item[Проучване] основни концепции при създаване на компютърна визуална симулация
				\item[Създаване на симулация] система, специфично създадена за нуждите на градската мрежа в град Пловдив. 
				Избор на технологии. Програмиране.
				\item[Провеждане на симулации] използване на реални данни като вход за създадената система
				\item[Резултати] обявяване на получените резултати			
			\end{description}		
			\item[Заключение] Наблюдения върху получените резултати и дискусия. 
			Постигнати ли са поставените цели и къде е имало проблеми. Какво може да бъде развито в бъдеще.
		\end{description}