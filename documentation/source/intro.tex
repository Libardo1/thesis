\chapter*{Въведение}\addcontentsline{toc}{chapter}{Въведение}\chaptermark{Въведение}

	Искали ли сте да прекарате повече време с детето си на закуска, 
	да се приберете по-рано от работа и да прекарате време с любимия човек или да излезнете с приятели? 
	Може и да поспите повече, разбира се. Колко хубаво би било това да е правило, вместо изключение.
	
	Всеки пловдивчанин, използващ пътната мрежа на града, губи средно около 60 мин. на ден в трафика.
	Времето, необходимо за достигане на квартал Тракия от Нова сграда на ПУ е 15 мин, при липса на задръствания.
	Това време се увеличава до 40 мин в периодите около 18:00-19:00 ч.
	Времето, прекарано в условия на трафик се отразява сериозно на здравето ни. \cite{HEI}
	Би било добре, това време да бъде прекарвано по друг, по-приятен начин.
	
	Към проблема е подхождано по множество различни начини, много от които предоставят реални и действащи решения.
	Генерално решение би подхождало при условие, че то може да бъде модифицирано в много голяма степен, така че да пасва на локалния проблем.
	Нуждата от локално и много специализирано решение, обаче, подсказва липса на генерално, добре пасващо такова.

	\subsubsection{Цел}

	Целта на дипломната работа е да предложи среда, в която да се изследват стратегии за намаляне на времето, 
	прекарано в пътната мрежа на град Пловдив. 

	\subsubsection{Изисквания}

		\begin{itemize}
			\item Цената за използването на градската мрежа трябва да остане същата или да е по-ниска.
			\item Повишаване на комфорта и спокойствието при използването на градската мрежа.
			\item Намаляне на вредните емисии във въздуха.
			\item Намаляне на броя катастрофи.
		\end{itemize}

	\subsubsection{Постигане на целта}

		За постигане на целите на дипломната работа, спрямо поставените изисквания, се разглеждат:
		\begin{itemize}
			\item Модерни технологични решения в подобни ситуации.
			\item Проучване на съществуващи симулационни системи.
			\item Създаване на симулация за намиране на критичните точки от транспортната мрежа и тяхното оптимизиране.
			\item Получените резултати се анализират, за да се получи частично или пълно решение на проблема.
		\end{itemize}

	\subsubsection{Целеви групи}
	
		Хора, които използват транспортната мрежа, през най-натоварените часове на денонощието. Важно за всеки от участниците е бързото достигане на съответна точка от града, без това да пречи на личното му здраве и комфорт.
	
	\subsubsection{Желан резултат}
	
		Дипломната работа може да се сметне за успешна, когато бъде създаден прототип на система, който да позволява 
		изследването и намалянето на трафика на града. Системата трябва да бъде лесна за промяна и разширяване,
		така че да предоставя реална подкрепа при разработване на начини за оптимизация на трафика.
		
	
	\subsubsection{Задачи}
		
		За постигане на желания резултат е необходимо следните задачи да бъдат изпълнени:	
	
		\begin{itemize}
			\item Проучване върху методите за изграждане на симулации
			\item Избор между съществуваща и специализирана за целта система
			\item Избор на програмен език
			\item Избор на технологии
			\item Програмиране на самата система
			\item Провеждане на експерименти
			\item Отчитане на резултатите от експериментите
			\item Препоръки за усъвършенстване на системата
		\end{itemize}