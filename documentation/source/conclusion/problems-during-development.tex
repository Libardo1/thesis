\subsubsection{Проблеми по време на разработка}

	\emph{Изказаните по-долу мнения са изцяло авторски и могат да бъдат разгледани като истина или некомпетентност от страна на автора.}
	
	За разработката на системата беше използвана \emph{Windows 7}\footnote{\url{http://windows.microsoft.com/en-US/windows/home}}
	базирана система. Липсата на мощни интрументи (по подразбиране) в командния ред, които да спомогнат бързата разработката,
	беше осезаема. Проблеми бяха срещнати и при настройка на кодирането(encoding).
	
	От друга страна, инсталациите на \emph{Lua} и \emph{Moai} бяха бързи и удобни. Редакторът, 
	\emph{Notepad++ (v6.1.2)}\footnote{\url{http://notepad-plus-plus.org/}}, 
	също предложи приятен и удобен интерфейс за писане на \emph{Lua}.

	\paragraph{Lua}
		Макар и разработван от дълго време, езикът има доста празнини, в сравнение с езици като
		\emph{C++} или \emph{Java}. Употребата му в големи компютърни игри не е довела до
		значително нарастване на помощните функции в стартовата библиотека. Напротив, повечето
		потребители на езика са запазили написаното като затворен код или просто не са отделили
		време да го споделят с останалите.
		
		За целите на симулацията, бяха необходими създаването на структури от данни, съществуващи
		в стандартните библиотеки на почти всеки друг програмен език от високо ниво. Макар и
		съществуващите \emph{таблици} да предлагат почти цялата функционалност, необходима
		за създаване на каквито и да е видове структури от данни, тя не е лесна за използване
		и преизползване, ако се ползва в ``суров`` вид.	Това налага нуждата от създаването на
		обвивки(wrappers).
		
		Липсата на стандартна библиотека за тестване, под какъвто и да е вид, направи използване на
		добър итеративен подход невъзможно. Още повече, това доведе до цялостно забавяне на процеса
		на разботка на системата.
		
	\paragraph{LuaRocks}
		
		\emph{LuaRocks}\footnote{\url{http://luarocks.org}} е добър опит да се направи лесно и бързо
		споделянето на \emph{Lua} код под формата на модули, наречени ``скали``(rocks). 
		Те съдържат информация за зависимостите(dependencies) които имат, чрез която гарантират
		наличие на всички необходими компоненти на системата.
		
		Инсталацията на \emph{LuaRocks} е бърза и лесна. Текущата версия \emph{(v2.0.8)}, обаче,
		просто не тръгва под \emph{Windows}. Проблемът се дължи в неправилно конфигуриране на пътищата
		из цялата програма.
		
		Липсата на \emph{LuaRocks} означава и липса на \emph{инсталирана} библиотека за тестване.
		Много обещаваща такава изглежда библиотеката за поведенческо тестване(behaviour driven testing) 
		\emph{luaspec}\footnote{\url{https://github.com/mirven/luaspec}}.
		
		Броят на пакетите в системата също не е впечатляващ, около 200 бр. За сравнение, разработчиците
		на \emph{Ruby} разполагат с над 2530 библиотеки\footnote{\url{https://rubygems.org/gems}}. 
		Те варират от малки с десетки редове код до огромни
		със стотици хиляди редове\footnote{\url{https://www.ohloh.net/p/rails/analyses/latest}}.
		
	\paragraph{Moai}
	
		След срещането на горепосочените проблеми, логичен избор би бил преминаването 
		към \emph{Linux} базирана система (напр. Ubuntu\footnote{\url{http://www.ubuntu.com/}}). 
		Това би решило, практически, всички тях.
		
		За съжаление, за моментa \emph{Moai} не поддържа лесна инсталация и разработка под Linux платформа.
		Тя работи прекрасно под \emph{Mac}, но такава система би излезнала ``малко`` 
		по-скъпо\footnote{Към момента на писането, най-евтиния вариант е 2395 лв. \url{http://bit.ly/JIawqv}}.
		
		По време на разработката на симулационната система беше използвано \emph{Moai SDK v1.1}.
		То идва с редица полезни примери, повечето от които имат неправилно зададени пътища, а други
		използват вече несъществуващи методи от по-стари версии.
		
		Документацията е полезна, но непълна. Целта на библиотеката е създаване на средно големи игри под всякакви платформи, 
		но примерен код за такива няма. За момента няма и издадени книги, които да предлагат подобни примери.