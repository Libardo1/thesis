\subsubsection{Резултати}

		За постигане на целите на дипломната работа е създадена примерна симулационна система. Тя е модуларна и разшируема
		чрез механизма за добавяне на нови стратегии с различна приоритизация на моделите. За създаването й е използван
		итеративен подход, като са добавени само най-необходимите функционалности. Това намалява сложността за доразвитие
		и улеснява разработката на нови възможности. 
		
		Употребата на \ac{OOP} прави лесно допълването на нови модели, начини за изобразяване на различните елементи и нови състояния.
		Разширяването се състои в изграждане на нови подкласове, отговарящи на определен интерфейс, които следват \ac{LSP}.
		
		Липсата на типове в \emph{Lua}, прави възможна динамичната замяна на конкретни имплементации за отделните стратегии, без
		необходимост от явни интерфейси.
		
		Създадената примерна имплементация е прототип на система, която може да бъде вградена
		в съвременната система за управление на транспорта в град Пловдив. Проведени са експерименти със стратегии 
		с ясно изразен приоритет на специфични модели. Те показват по-голям коефициент на ефективност и превозени пътници за единица време.
		За да бъдат приети или отхвърлени, получените резултати трябва да се сравнят с данни от реални експерименти.